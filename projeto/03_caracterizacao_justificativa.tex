\section{Caracterização e Justificativa}
\label{sec:caracterizacao_justificativa}

Diante do exposto, este projeto tem a seguinte questão de pesquisa: \textit{\lipsum[20]?}


% um modelo de clusterização de \textit{codelets} utilizando Gaussian Mixture Model (GMM) e Algoritmo Genético (GA). Usamos o GA para seleção de \textit{features} usando o índice de silhueta como função objetivo. O restante deste artigo está estruturado da seguinte forma:\autoref{subsec:trabalhos_relacionados} apresenta uma breve revisão dos trabalhos relacionados. \autoref{sec:methods} apresenta os métodos e configurações usados nos experimentos. Em \autoref{sec:results_discussion}, apresentamos e discutimos os resultados. Finalmente, \autoref{sec:conclusões} conclui o estudo.

% Diante do exposto, este projeto tem a seguinte questão de pesquisa:\textit{Dado um programa, ou uma parte computacionalmente intensiva deste, e um sistema de hardware heterogêneo, contendo plataformas com arquiteturas diferentes, quais características geradas por \textit{profiling} em \textit{benchmarks} sintéticos nestes sistemas são relevantes para otimização desempenho e de consumo de energia?}

Exposto o problema, segue-se para a revisão bibliográfica dos temas que embasarão a pesquisa proposta. O estado da arte foi levantando a partir de pesquisas em bancos de artigos científicos utilizando palavras-chave como:\textit{palavra-chave 1}, \textit{palavra-chave 2}, \textit{palavra-chave 3}, \textit{palavra-chave 4}, \textit{palavra-chave 5} e  \textit{palavra-chave 6}.

\subsection{\textit{Tópico de revisão 1}} \label{subsec:topico_revisao_1}

\lipsum[30-33]~\cite{inbook,proceedings}.

\subsection{\textit{Tópico de revisão 2}} \label{subsec:topico_revisao_2}

\lipsum[40-43]~\cite{book}.

\subsection{Trabalhos correlatos}\label{subsec:trabalhos_correlatos}

\citet{article} apresentam \lipsum[38].

\citet{manual} apresentam \lipsum[39].


\subsection{Justificativa}

Justifica-se este estudo...\lipsum[50]